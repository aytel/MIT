\documentclass[12pt]{article} 

\usepackage{cmap} 
\usepackage[T2A]{fontenc}
\usepackage[russian,english]{babel}
\usepackage[utf8]{inputenc} 
\usepackage{amsmath, amssymb} 

\usepackage{hologo} 
\usepackage[russian]{hyperref} 

\textwidth=160mm
\hoffset=-15mm
\textheight=240mm
\voffset=-20mm

\newcommand{\Section}[1]{\section{#1}\vspace{-1.5em}\hspace*{\parindent}\unskip} 

\begin{document}
\def\t{\texttt}

Бабушкин А.

\section{Обязательные задачи}

\begin{enumerate}
    \setlength{\parskip}{0pt} 
    \setlength{\itemsep}{0pt} 
    \item ~\\
    \item ~\\
    a) Сделаем его на персистентным массиве, будет $n\log n$ памяти и $n\log^2 n$ времени. \\
    b) Построим и обойдём дерево версий. Нам нужен будет СНМ с откатами, поэтому в сжатии путей смысла нет. nlog памяти и времени. \\
    \item ~\\
    \item ~\\
    \item Построим центроиду. Теперь все запросы разбились на два запроса "на пути от корня дерева до вершины среди префикса рёбер по весу узнать 
    максимум цены". Это похоже на обычный запрос к ДД, где ключом будет вес, но непонятно как такое ДД строить. Давайте сделаем его персистентным -- 
    пока будем на очередном шаге обходить поддерево центроиды, будем закидывать новое ребро в персистентное ДД и получать новую версию ДД, Построим 
    мы это всё за $n\log^2 n$, но отвечать на запрос будем за лог -- найти разделяющую вершины центроиду и сделать два запроса к ДД.\\
\end{enumerate}

\section{Дополнительные задачи}

\begin{enumerate}
    \setlength{\parskip}{0pt} 
    \setlength{\itemsep}{0pt} 
    \item 
\end{enumerate}

\end{document}
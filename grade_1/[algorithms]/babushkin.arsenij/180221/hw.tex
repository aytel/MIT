\documentclass[12pt]{article} 

\usepackage{cmap} 
\usepackage[T2A]{fontenc}
\usepackage[russian,english]{babel}
\usepackage[utf8]{inputenc} 
\usepackage{amsmath, amssymb} 

\usepackage{hologo} 
\usepackage[russian]{hyperref} 

\textwidth=160mm
\hoffset=-15mm
\textheight=240mm
\voffset=-20mm

\newcommand{\Section}[1]{\section{#1}\vspace{-1.5em}\hspace*{\parindent}\unskip} 

\begin{document}
\def\t{\texttt}

Бабушкин А.

\section{Обязательные задачи}

\begin{enumerate}
	\setlength{\parskip}{0pt} 
	\setlength{\itemsep}{0pt} 
	\item Чтобы добиться ошибки $e^{-1}$, надо повторить $n^3$ раз, а чтобы затем $1 / {2^n}$ -- ещё n. Итого $n^4$ повторов, а всё 
    решение за $n^7$. \\
    \item Проверим Миллером-Рабином, простое ли число. Если оно составное, то затем тыкнем в случайное число, взаимно просто с n, и 
    проверим, что оно свидетель простоты для Ферма. \\
    Если число простое, то Миллером-Рабином мы это найдём. \\
    Если составное и не Кармайкла, то с верояностью $\le 1/2$ ошибётся Миллер-Рабин, и мы его не найдём. И ещё с верояностью $\le 1/2$ 
    ошибётся Ферма и выберет свидетеля простоты. \\
    Если составное и Кармайкла, то может ошибиться только Миллер-Рабин, а Ферма точно скажет, что оно простое. \\
    Ошибка двусторонняя, но $\log n$ проверок всё так же хватит. \\
    \item Будем выбирать случайные y и вычислять g(x + y) - g(y), затем среди всех результатов выберем самый частый. Вероятность ошибиться 
    у нас $2\epsilon - \epsilon^2$, будем понижать её, повторяя много раз. \\
    \item Если в каждом клозе три различных переменных, то он выполняется с вероятностью 7/8, тогда матожидание числа выполненных -- 
    7m/8. Если мы хотим выполнить $\ge$ 3m/4 $=$ 6m/8 клозов, то надо оступить от матожидания не больше чем на m/8, тогда хватит m 
    запусков. \\
    \item Если ответ нет, то RP не сможет ответить да. Если ответ да, то включим в подсказку случайные биты, на которых RP отвечает правильно. \\
    \item На всех C1 тестах корректно Поллард отработает с верояностью $(1 - 1/2)^{C1}$, а если повторить это k раз, то хотя бы раз 
    победит с верояностью $1 - (1 - (1/2)^{C1})^k$. $(1 - (1/2)^{C1})^k$ должно быть $\le C2$, то есть $k \ge \log _{1 - (1/2)^{C1}}C2$. \\
\end{enumerate}

\section{Дополнительные задачи}

\begin{enumerate}
	\setlength{\parskip}{0pt} 
	\setlength{\itemsep}{0pt} 
	\item Пусть у них есть графы G и H, и они проведут несколько итераций таких действий: \\
    1. Алиса выбирает какой-то граф M, который изоморфен G, а значит, и H, и передаёт его Бобу. \\
    2. Боб выбирает случайный бит, и если он равен 0, то Алиса сообщает Бобу перестановку для изоморфизма G -> M, иначе перестановку 
    для изоморфизма H -> M. Боб может легко проверить, сказала ли она правду. 
    Она сможет ответить на оба эти запроса только в том случае, если G и H действительно изоморфны. \\
    \item 
\end{enumerate}

\end{document}
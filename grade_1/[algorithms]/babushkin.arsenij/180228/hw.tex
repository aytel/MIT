\documentclass[12pt]{article} 

\usepackage{cmap} 
\usepackage[T2A]{fontenc}
\usepackage[russian,english]{babel}
\usepackage[utf8]{inputenc} 
\usepackage{amsmath, amssymb} 

\usepackage{hologo} 
\usepackage[russian]{hyperref} 

\textwidth=160mm
\hoffset=-15mm
\textheight=240mm
\voffset=-20mm

\newcommand{\Section}[1]{\section{#1}\vspace{-1.5em}\hspace*{\parindent}\unskip} 

\begin{document}
\def\t{\texttt}

Бабушкин А.

\section{Обязательные задачи}

\begin{enumerate}
	\setlength{\parskip}{0pt} 
	\setlength{\itemsep}{0pt} 
	\item ~\\
    a) Поделим на k, решим 1-2, домножим ответ на k. \\
    b) У нас будут очереди, и в i-ой очереди лежат вершины с расстоянием из [ik, (i + 1)k). Дальше как в 1-2. \\
    \item Давайте сделаем m/n-ичную кучу. \\
    \item Запихаем все вершины из А в очередь, проставив им расстояние 0, и запустим bfs. Так мы найдём кратчайшее расстояние, а количество 
    путей потом считаем динамикой на ДАГе, оставив только те рёбра a-b, для которых d[a] + 1 = d[b]. \\
    \item Запустим Флойда. Ответ на запрос -- ребро s-t веса w, хорошее, если d[a][b] = d[a][s] + w + d[t][b] или d[a][b] = d[a][t] + w + 
    d[s][b]. \\
    \item Запустим Дейкстру только на v и вершинах из А. Оставим только полезные рёбра, посчитаем на ДАГе сколько есть кратчайших путей из v 
    в каждую вершину. Затем сделаем то же самое на всех вершинах. Если у вершины количество кратчайших путей в первом случае и во втором не равно, 
    то она хорошая. \\
    \item Назовём состоянием пару вершин (s, t), то есть первый стоит в s, второй -- в t. Нам надо попасть из (v, u) в (u, v). 
    Запустим bfs по состояниям. Почему работает nm? Для ребра a-b мы рассмотрим его только в тех состояниях, где одна из вершин или a, или b, 
    а вторая вершина может быть любой, то есть в O(n) состояниях. Успех. \\
\end{enumerate}

\section{Дополнительные задачи}

\begin{enumerate}
	\setlength{\parskip}{0pt} 
	\setlength{\itemsep}{0pt} 
	\item Запустим от s Дейкстру, которая минимизирует два наибольших ребра на пути, запустим такую же по обратным рёбрам из t. \\
    Теперь переберём вершину v. Пусть первая Дейкстра дала пару (a, b), а вторая -- (c, d). Тогда если b <= c и d <= a, то обновим ответ 
    значением a + c. Почему работает? Рассмотрим оптимальный путь. На нём точно есть вершина, которая лежит между двух самых тяжёлых рёбер. 
    Тогда в такой вершине будет правильный ответ. \\
\end{enumerate}

\end{document}
\documentclass[12pt]{article} 

\usepackage{cmap} 
\usepackage[T2A]{fontenc}
\usepackage[russian,english]{babel}
\usepackage[utf8]{inputenc} 
\usepackage{amsmath, amssymb} 

\usepackage{hologo} 
\usepackage[russian]{hyperref} 

\textwidth=160mm
\hoffset=-15mm
\textheight=240mm
\voffset=-20mm

\newcommand{\Section}[1]{\section{#1}\vspace{-1.5em}\hspace*{\parindent}\unskip} 

\begin{document}
\def\t{\texttt}

Бабушкин А.

\section{Обязательные задачи}

\begin{enumerate}
    \setlength{\parskip}{0pt} 
    \setlength{\itemsep}{0pt} 
    \item Построить AVL на отрезке -- это взять середину отрезка и сделать корнем, после этого построить AVL на левой и правой половине и 
    их корни присоединить к себе. \\
    \item Держим AVL с теми же ключами, что и в нашем BST, ещё там лежат ссылки на соответствующие вершины в BST. Хотим добавить x. Находим 
    lower_bound(x) в AVL и prev(lower_bound(x)). Смотрим их вершины в BST. Добавляем х правым сыном prev или левым сыном lower_bound (только одно из 
    этих мест свободно). Добавляем x в AVL. \\
    \item Для вершины в AVL держим minx, miny, maxx, maxy в поддереве. Ответ на запрос -- обходим аналогично задачам из практики, добываем 
    настоящие границы по x и y, выводим ответ. \\
    \item ~\\
    \t{
        Node\* lower\_bound(Node\* v, int x) \{ \\
        if (v == NULL) \\
                return v; \\
            if (v->x < x) \\
                return lower\_bound(v->r, x); \\
            if (v->x >= x) \{ \\
                fnd = lower\_bound(v->l, x); \\
                return (fnd ? fnd : v); \\
            \} \\
        \}
    } \\
    \item Умеем искать минимум y на отрезке за лог -- нашли, удалили, повторили k раз. Потом запихнём всё обратно. \\
    \item Дерево (y, x) по ключу y. Умеем искать минимум x на отрезке за лог -- нашли, удалили, и так пока минимум не стал больше r. Потом вернули. \\
\end{enumerate}

\section{Дополнительные задачи}

\begin{enumerate}
    \setlength{\parskip}{0pt} 
    \setlength{\itemsep}{0pt} 
    \item ~\\
    \item Извлечём n / $\log n$ раз $\log n$ минимальных со всего отрезка. После этого посортим каждый блок за $\log n \log \log n$. \\
    Итого посортили все y за $n \log \log n$. Ой. \\
\end{enumerate}

\end{document}
%
% Проверьте, что вы сохранили файл в кодировке utf8
%

\documentclass[12pt]{article} % 12 -- размер шрифта

\usepackage{cmap} % Чтобы можно было копировать русский текст из pdf
\usepackage[T2A]{fontenc}
\usepackage[russian,english]{babel}
\usepackage[utf8]{inputenc} % Проверьте, что кодировка файла -- тоже utf8
\usepackage{amsmath, amssymb} % Чтобы юзать математические символы

\usepackage{hologo} % Логотип LaTex
\usepackage[russian]{hyperref} % http ссылки на внешние источники

% Меняем размер листа, можно не менять
\textwidth=160mm
\hoffset=-15mm
\textheight=240mm
\voffset=-20mm

\newcommand{\Section}[1]{\section{#1}\vspace{-1.5em}\hspace*{\parindent}\unskip} % Фиксим багу с отступом в начале section, заодно пример собственных функций

\begin{document}

Бабушкин А.

\begin{enumerate}
  \setlength{\parskip}{0pt} % Отступ перед списком
  \setlength{\itemsep}{0pt} % Отступ между строками списка
  \item ~\\
	l) $\log(n^{\log n}) = \log n * \log \log n < \log^{2} n < n = \log (1.1^{n}) \to $ True. \\
	m) $\frac {n^3}{n^2 + n \log n} = \frac {n^3}{n^2 + o(n^2)} = \frac {n^3}{\Theta(n^2)} = \Theta (\frac {n^3}{n^2}) = \Theta ({n}) < {O} (n \log n) \to $ True. \\
	n) Пусть $f(n) = 2^n$. Тогда $f(n) = (f({\frac{n}{2}}))^2, x \ne O(x^2) \to$ False. \\
	o) $f(n) - o(f(n)) \le f(n) \to f(n) - 0(f(n)) = O(f(n)) \\
 f(n) - o(f(n)) \ge f(n) - \frac {f(n)}{C} =  \frac {1 - C}{C} f(n) \to \frac {C}{1 - C}(f(n) - o(f(n)) \ge f(n) \to f(n) - o(f(n)) = \Omega(f(n)) \to $ True. Для $f(n) + o(f(n))$ аналогично.\\
	p) $2^{\log n!} = n! < n^n = 2^{n \log n} \to \log n! < n \log n \to \log n! = O(n \log n)$. \\
	Теперь докажем $\Omega$. $\log n! = \sum {\log i} : i \le n$. Возьмём первые $\frac {n}{2}$ из них. Каждое из слагаемых не меньше $\log \frac {n}{2} = \log (n - 1)$, то есть их сумма $\ge \frac {n}{2} (\log n - 1) = \Theta (n \log n) \to $ True.
  \item ~\\
	g) Докажем, что $T(n) = \Theta(n)$ по индукции. База очевидна, теперь переход. Пусть это верно $\forall i < n$. Сначала докажем $O$. $T(n) = T(\frac {n}{2}) + T(\frac {n}{3}) + n \le C\frac{n}{2} + C\frac {n}{3} + n = \frac {(5C + 6)}{6}n $. Если $\exists C : \frac {5C + 6}{6}  \le C$, то мы победили. Это равенство верно для $C \ge 6 \to$ True. $\Omega$ доказывается точно так же, только там нужно будет $\exists C : \frac {5C + 6}{6} \ge C \to C \le 6$. \\
	h) $T(n) = \Theta(n^2)$ по мастер-теореме ($a= 4, b = 2, c = 1, d = 2$). \\
	i) $T(n) = \Theta(n^{\log_{3}{2}})$ по мастер-теореме ($a= 2, b = 3, c = 0, d = 0$). \\
	j) $T(n) = \Theta(n^{\frac {\sqrt{5} + 1}{2}})$, т.к. это числа Фибоначчи. \\
	k) $T(n) = T(n - 1) + n \to T(n) = \frac {n(n + 1)}{2} = \Theta(n^2)$.
  \item ~\\
 $\begin{array}{|cc|c|c|c|c|c|}
    \hline
    A & B & O & o & \Theta & \omega & \Omega \\
    \hline
    n & n^2 & + & + & - & - & - \\
    \log^k n & n^{\epsilon} & + & + & - & - & - \\
    n^k & c^n & + & + & - & - & - \\
    \sqrt{n} & n^{\sin n} & - & - & - & - & - \\
    2^n & 2^{n \slash 2} & - & - & - & + & + \\
    n^{\log m} & m^{\log n} & + & - & + & - & + \\
    \log (n!) & \log(n^n) & + & - & - & - & + \\
    \hline
  \end{array}$
  \item Разобьём на классы эквивалентности (если функции вместе, то они $\Theta$ друг от друга) и упорядочим по возрастанию.\\
	\begin{enumerate}
	  \setlength{\parskip}{0pt} % Отступ перед списком
	  \setlength{\itemsep}{0pt} % Отступ между строками списка
	  \item $1, n^{1/\log n}$ \\
	  \item $\log (\log^* n)$ \\
	  \item $\log^* n, \log^* \log n$ \\
	  \item $2^{\log^* n}$ \\
	  \item $\ln \ln n$ \\
	  \item $\sqrt{\log n}$ \\
	  \item $\ln n$ \\
	  \item $\log^2 n$ \\
	  \item $2^{\sqrt{2 \log n}}$ \\
	  \item $n, 2^{\ln n}$ \\
	  \item $n \log n, \log n!$ \\
	  \item $n^2, 4^{\log n}$ \\
	  \item $n^3$ \\
	  \item $(\log n)!$ \\
	  \item $n^{\log \log n}, \log n^{\log n}$ \\
	  \item $ (\sqrt n)^{\log n} $ \\
	  \item $(\frac{3}{2})^n$ \\
	  \item $2^n$ \\
	  \item $n * 2^n$ \\
	  \item $e^n$ \\
	  \item $n!$\\
	  \item $(n + 1)!$ \\
	  \item $2^{2^n}$ \\
	  \item $2^{2^{n+1}}$ \\
	\end{enumerate}
  \item Сумма бесконечной геометрической прогрессии $ = \frac {b}{1 - q}$.\\
	a) $b = 1, q = \frac {1}{2} \to$ ответ $ = 2$. \\
	b) Это сумма двух прогрессий. У одной $b = 1, q = \frac {1}{4}$, у другой $b = -\frac{1}{2}, q = \frac{1}{4} \to $ ответ $ = \frac {2}{3}$. \\

\end{enumerate}

\end{document}

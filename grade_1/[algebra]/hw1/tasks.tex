\documentclass[12pt]{article}

\usepackage{cmap}
\usepackage[T2A]{fontenc}
\usepackage[utf8]{inputenc}
\usepackage[russian]{babel}
\usepackage{graphicx}
\usepackage{amsthm,amsmath,amssymb}
\usepackage[russian,colorlinks=true,urlcolor=red,linkcolor=blue]{hyperref}
\usepackage{enumerate}
\usepackage{datetime}
\usepackage{minted}
\usepackage{fancyhdr}
\usepackage{lastpage}
\usepackage{color}
\usepackage{verbatim}

\def\NAME{Асимптотика и рекуррентные соотношения}
\def\DATE{6 сентября}
\def\CURNO{\NO\t{1}}

\parskip=0em
\parindent=0em

\sloppy
\voffset=-20mm
\textheight=235mm
\hoffset=-25mm
\textwidth=180mm
\headsep=12pt
\footskip=20pt

\setcounter{page}{0}
\pagestyle{empty}

% Основные математические символы
\DeclareSymbolFont{extraup}{U}{zavm}{m}{n}
\DeclareMathSymbol{\heart}{\mathalpha}{extraup}{86}
\newcommand{\N}{\mathbb{N}}   % Natural numbers
\newcommand{\R}{\mathbb{R}}   % Ratio numbers
\newcommand{\Z}{\mathbb{Z}}   % Integer numbers
\def\EPS{\varepsilon}         %
\def\SO{\Rightarrow}          % =>
\def\EQ{\Leftrightarrow}      % <=>
\def\t{\texttt}               % mono font
\def\c#1{{\rm\sc{#1}}}        % font for classes NP, SAT, etc
\def\O{\mathcal{O}}           %
\def\NO{\t{\#}}               % #
\renewcommand{\le}{\leqslant} % <=, beauty
\renewcommand{\ge}{\geqslant} % >=, beauty
\def\XOR{\text{ {\raisebox{-2pt}{\ensuremath{\Hat{}}}} }}
\newcommand{\q}[1]{\langle #1 \rangle}               % <x>
\newcommand\URL[1]{{\footnotesize{\url{#1}}}}        %
\newcommand{\sfrac}[2]{{\scriptstyle\frac{#1}{#2}}}  % Очень маленькая дробь
\newcommand{\mfrac}[2]{{\textstyle\frac{#1}{#2}}}    % Небольшая дробь
\newcommand{\score}[1]{{\bf\color{red}{(#1)}}}

% Отступы
\def\makeparindent{\hspace*{\parindent}}
\def\up{\vspace*{-0.3em}}
\def\down{\vspace*{0.3em}}
\def\LINE{\vspace*{-1em}\noindent \underline{\hbox to 1\textwidth{{ } \hfil{ } \hfil{ } }}}
%\def\up{\vspace*{-\baselineskip}}

\lhead{Алгоритмы, осень 2017/18}
\chead{}
\rhead{Практика \CURNO. \NAME.}
\renewcommand{\headrulewidth}{0.4pt}

\lfoot{}
\cfoot{\thepage\t{/}\pageref*{LastPage}}
\rfoot{}
\renewcommand{\footrulewidth}{0.4pt}

\newenvironment{MyList}[1][4pt]{
  \begin{enumerate}[1.]
  \setlength{\parskip}{0pt}
  \setlength{\itemsep}{#1}
}{       
  \end{enumerate}
}
\newenvironment{InnerMyList}[1][0pt]{
  \vspace*{-0.5em}
  \begin{enumerate}[a)]
  \setlength{\parskip}{#1}
  \setlength{\itemsep}{0pt}
}{
  \end{enumerate}
}

\newcommand{\Section}[1]{
  \refstepcounter{section}
  \addcontentsline{toc}{section}{\arabic{section}. #1} 
  %{\LARGE \bf \arabic{section}. #1} 
  {\LARGE \bf #1} 
  \vspace*{1em}
  \makeparindent\unskip
}
\newcommand{\Subsection}[1]{
  \refstepcounter{subsection}
  \addcontentsline{toc}{subsection}{\arabic{section}.\arabic{subsection}. #1} 
  {\Large \bf \arabic{section}.\arabic{subsection}. #1} 
  \vspace*{0.5em}
  \makeparindent\unskip
}

% Код с правильными отступами
\newenvironment{code}{
  \VerbatimEnvironment

  \vspace*{-0.5em}
  \begin{minted}{c}}{
  \end{minted}
  \vspace*{-0.5em}

}

% Формулы с правильными отступами
\newenvironment{smallformula}{
 
  \vspace*{-0.8em}
}{
  \vspace*{-1.2em}
  
}
\newenvironment{formula}{
 
  \vspace*{-0.4em}
}{
  \vspace*{-0.6em}
  
}

\definecolor{dkgreen}{rgb}{0,0.6,0}
\definecolor{brown}{rgb}{0.5,0.5,0}
\newcommand{\red}[1]{{\color{red}{#1}}}
\newcommand{\dkgreen}[1]{{\color{dkgreen}{#1}}}

\begin{document}

\renewcommand{\dateseparator}{--}
\begin{center}
  {\Large\bf 
   Первый курс, осенний семестр 2017/18\\
   Практика по алгоритмам \CURNO\\
   \vspace*{0.5em} 
   \NAME \\
   \vspace*{0.5em} 
   \DATE}\\
  \vspace{0.5em}
  {Собрано {\today} в {\currenttime}}
\end{center}

\vspace{-1em}
\LINE
\vspace{1em}

%\tableofcontents

%\pagebreak
\pagestyle{fancy}

\vspace*{0.5em}
\Section{Домашнее задание}

\Subsection{Обязательная часть}

\begin{MyList}[6pt]

  \item \score{2.5} {\bf Истина или ложь?}, \score{0.5} за каждый пункт\\
    Проверьте корректность, докажите.
    \begin{InnerMyList}
      \setcounter{enumii}{11}
      \item $n^{\log n} = \O(1.1^n)$
      \item $\frac{n^3}{n^2 + n \log n} = \O(n \log n)$
      \item $f(n) = \O(f(\frac{n}{2}))$
      \item $f(n) \pm o(f(n)) = \Theta(f(n))$
      \item $\log n! = \Theta(n \log n)$
    \end{InnerMyList}

  \item \score{2.5} {\bf Рекуррентность}, \score{0.5} за каждый пункт\\
    Во всех задачах предполагается, что $\forall x \le 1, T(x) = 1$
    \begin{InnerMyList}
      \setcounter{enumii}{6}
      \item $T(n) = T(\mfrac{n}{2}) + T(\mfrac{n}{3}) + n$
      \item $T(n) = 4T(\mfrac{n}{2}) + n \log^2 n$
      \item $T(n) = 2T(\mfrac{n}{3}) + 1$
      \item $T(n) = T(n - 1) + T(n - 2)$
      \item $T(n) = T(n - 1) + n$
    \end{InnerMyList}

  \item \score{3} {\bf Заполнить табличку}\\
    $A = O(B)$? $A = o(B)$? и т.д.
    За каждый неправильный ответ \score{${-}0.1$} балл.
$$
  \begin{array}{|cc|c|c|c|c|c|}
    \hline
    A & B & O & o & \Theta & \omega & \Omega \\
    \hline
    n & n^2 & + & + & - & - & - \\
    \log^k n & n^{\epsilon} & & & & & \\
    n^k & c^n & & & & & \\
    \sqrt{n} & n^{\sin n} & & & & & \\
    2^n & 2^{n \slash 2} & & & & & \\
    n^{\log m} & m^{\log n} & & & & & \\
    \log (n!) & \log(n^n) & & & & & \\
    \hline
  \end{array}
$$
  \item \score{6} {\bf Упорядочить 30 функций в порядке возрастания}\\
    Если какие-то функции равны ($f = \Theta(g)$), указать это.
    Здесь $\log n$ --- двоичный логарифм, $\ln n$ --- натуральный логарифм.
    За каждый неправильный ответ \score{${-}0.2$} балла.

$$
  \begin{array}{cccccc}
    \log(\log^* n) & 2^{\log^* n} & (\sqrt{n})^{\log n} & n^2 & n! & (\log n)! \\
    (3 \slash 2)^n & n^3 & \log^2 n & \log n! & 2^{2^n} & n^{1 \slash \log n} \\
    \ln \ln n & \log^* n & n \cdot 2^n & n^{\log \log n} & \ln n & 1 \\
    2^{\ln n} & (\log n)^{\log n} & e^n & 4^{\log n} & (n + 1)! & \sqrt{\log n} \\
    \log^* \log n & 2^{\sqrt{2 \log n}} & n & 2^n & n \log n & 2^{2^{n + 1}}          
  \end{array}
$$
    Примечание: $\log^*(n) = \left\{
    \begin{array}{ll}
        0 & \text{ если } n \leq 1;\\
        1 + \log^*(\log n) & \text{ иначе}
    \end{array}
    \right.$

%\pagebreak
  \item \score{1} {\bf Посчитать точно}, \score{0.5} за каждый пункт
    \begin{InnerMyList}
      \item $\sum_{k = 0}^\infty \frac{1}{2^k}$ 
      \item $\sum_{k = 0}^\infty \frac{(-1)^k}{2^k}$
    \end{InnerMyList}

\end{MyList}

%*\vspace{0.5em}
\pagebreak
\vspace*{0.5em}
\Subsection{Дополнительная часть}

\begin{MyList}[8pt]

  \item \score{1} {\bf Истина или ложь?}, \score{0.5} за каждый пункт\\
    Проверьте корректность, докажите.
    \begin{InnerMyList}
      \setcounter{enumii}{16}
      \item $n^n = \O(n!)$
      \item $n \log n - \log n! = \Theta(n)$
    \end{InnerMyList}

  \item \score{1.5} {\bf Рекуррентность}, \score{0.5} за каждый пункт\\
    Во всех задачах предполагается, что $\forall x \le 1, T(x) = 1$.
    \begin{InnerMyList}
      \setcounter{enumii}{11}
      \item $T(n) = T(n - 1) + T(n - 2) + 1$
      \item $T(n) = T(n - 1) + T(n - 3)$
      \item $T(n) = T(\mfrac{n}{2}) \cdot T(\mfrac{n}{3})$, при этом $\forall x \le 3 \ T(x) = 2$
    \end{InnerMyList}

  \item \score{1} {\bf Реккурентность и практика}\\
    Дайте наиболее точный ответ, докажите наиболее точные верхние и нижние оценки для
    $T(n) = T(n-1) + T(n^{1/3})$.

\end{MyList}

\end{document}

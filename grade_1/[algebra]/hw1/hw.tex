%
% Проверьте, что вы сохранили файл в кодировке utf8
%

\documentclass[12pt]{article} % 12 -- размер шрифта

\usepackage{cmap} % Чтобы можно было копировать русский текст из pdf
\usepackage[T2A]{fontenc}
\usepackage[russian,english]{babel}
\usepackage[utf8]{inputenc} % Проверьте, что кодировка файла -- тоже utf8
\usepackage{amsmath, amssymb} % Чтобы юзать математические символы

\usepackage{hologo} % Логотип LaTex
\usepackage[russian]{hyperref} % http ссылки на внешние источники

\newcommand{\Section}[1]{\section{#1}\vspace{-1.5em}\hspace*{\parindent}\unskip} % Фиксим багу с отступом в начале section, заодно пример собственных функций

\begin{document}

Бабушкин А.

\begin{enumerate}
  \setlength{\parskip}{0pt} % Отступ перед списком
  \setlength{\itemsep}{0pt} % Отступ между строками списка
  \item  Сначала посчитаем f(x):\\
x:~~~ 1 2 3 4 5 6 7 8 9 10\\
f(x): 1 2 3 3 4 5 5 6 7 7 \\
	а) Если $gf = f,$ то для  $x \in E_{f}, g(x) = x,$ а для остальных -- любое значение. Таких остальных $3$, следовательно, ответ $10^3 = 1000$.\\
	б) $g(f(x)) = g(x) \to $  все  $x$ образуют классы эквивалентности по $g(x)$, причём $x$ и $f(x)$ точно лежат в одном. Таких классов $3: (1), (2), (3...10)$, что следует из $f$ ($f(4) = 3, f(5) = 4,$ и т.д.), и у каждого класса своё значение $g(x)$, которое может быть любым. Тогда ответ снова $10^3 = 1000$.  \\
	в) $fg = f \to$ те $x$, которые имеют равные значения $f(x)$, могут переходить друг в друга в $g$. $f(x)$ имеет четыре значения, имеющих только один прообраз ($1, 2, 4, 6$), и три значения, имеющих два прообраза ($3, 5, 7$). Внутри классов прообразов отображения могут быть какими угодно. Первые четыре класса прообразов состоят из одного элемента, который должен переходить в себя ($g(1) = 1, g(2) = 2, g(5) = 5, g(8) = 8$), а оставшиеся три дают четыре варианта каждая, например, $g(3) = 3$ или $g(3) = 4$, и аналогично для $4, 6, 7, 9, 10$. Ответ $2^6 = 64$. \\
	г) $fg = g \to \forall x \in E_{g} ~ f(x) = x \to E_{g} \subset \{1, 2, 3\}$, т.к. только в $1, 2, 3 ~ f(x)$ имеет петли. Ответ $3^{10} = 59049$.
  \item ~\\
	а) Пусть $f(x) = i$, тогда давайте в ориентированном пути вершина под номером $x$  будет $i-$ой по порядку (рёбра идут от $i-$ой к $i+1-$ой вершине). Тогда, очевидно, разным перестановкам соотвествуют разные пути (если пути различны, то они отличаются хотя бы в одной позиции) $\to$ инъекция есть, и для каждого пути будет перестановка $\to$ сюръекция есть. \# \\
	б) \\
	в)  Кол-во отображений $f: M \to M = n^n$. Если оно равно количеству деревьев с двумя фиксированными вершинами, то количество деревьев без фиксирования вершин равно $ \frac {n^n}{C^{2}_{n}} = n^{n - 2}$.\\
  \item Если $f: M \to M$ -- биекция, то она состоит из набора непересекающихся циклов. По условию все циклы нечётной длины. Пусть $x$ лежит на цикле длины $l$. Тогда назначим $g(x) = f^{\frac{l + 1}{2}}(x)$. Тогда $g^{2}(x) = f^{l + 1}(x) = f(x)$. \#
  \item Пусть в $f$ существует цикл длины $k$. Докажем, что при применении $g$ он полностью переходит в цикл (тоже находящийся в $f$) длины $k$. $g(f(x)) = f(g(x))$, то есть $g(x)$ и $g(f(x))$ лежат на одном цикле по $f$. Докажем, что $f^{k}(g(x)) = g(f^{k}(x))$. Для $k = 1$ это верно. Пусть верно для $k$, докажем для $k + 1$. $f^{k + 1}(g(x)) = f(f^{k}(g(x)) = f(g(f^{k}(x))) = g(f(f^{k}(x))) = g(f^{k + 1}(x))$. \# Теперь можно увидеть, что если мы будем увеличивать $k$, то мы как раз пойдём параллельно по двум циклам в $f$ -- один до применения $g$, другой после. Таким образом, все циклы переходят в циклы той же длины. \# \\
Осталось только научиться считать ответ. Пусть у нас есть $x$ циклов длины $y$. Тогда они должны переходить друг в друга, таких вариантов $x\!$. Но для каждого цикла, переходящего в цикл, надо ещё выбрать сдвиг ($(1, 2, 3) \to (1, 2, 3), \to (2, 3, 1), $~или~$ \to (3, 2, 1)$). Для каждого цикла $y$ вариантов циклических сдвигов, всего вариантов $x! * y^x$, и это надо перемножить по всем $y$. \\
	а) Здесь есть один цикл длины $16$, ответ $1! * 16 = 16$. \\
	б) Табличка для $f(x)$: \\
x:~~~ 0 1 2 3 4 5 6 7 8 9 10 11 12 13 14 15 \\
f(x): 0 3 6 9 12 15 2 5 8 11 14 1 4  7 10 13
Здесь есть такие циклы: \\
Длины 1: $(0 \to 0), (8 \to 8)$ \\
Длины 2: $(10 \to 14 \to 10), (2 \to 6 \to 2), (4 \to 12 \to 4)$ \\
Длины 4: $(1 \to 3 \to 9 \to 11 \to 1), (5 \to 15 \to 13 \to 7 \to 5)$ \\
Значит, ответ $(2! * 1^2) * (3! * 2^3) * (2! * 4^2) = 3 * 2^10 = 3072$. \\
	в) Здесь $4$ цикла длиной $4$, ответ $4! * 4^4 = 6144$.
\end{enumerate}

\end{document}

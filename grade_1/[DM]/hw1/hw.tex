\documentclass[12pt]{article} 

\usepackage{cmap} 
\usepackage[T2A]{fontenc}
\usepackage[russian,english]{babel}
\usepackage[utf8]{inputenc} 
\usepackage{amsmath, amssymb} 

\usepackage{hologo} 
\usepackage[russian]{hyperref} 

\textwidth=160mm
\hoffset=-15mm
\textheight=240mm
\voffset=-20mm

\newcommand{\Section}[1]{\section{#1}\vspace{-1.5em}\hspace*{\parindent}\unskip} 

\begin{document}

Бабушкин А.

\begin{enumerate}
	\setlength{\parskip}{0pt} 
	\setlength{\itemsep}{0pt} 
	\item Сначала оценим. В каждом квадрате $2 \times 2$ может стоять не больше одного короля, а доска разбивается на $16$ таких квадратов. Тогда ответ $\le 16$. Приведём пример на $16$. Для этого просто поставим королей во все клетки, чей столбец нечётный, как и порядковый номер буквы, бить они друг друга не будут. Ответ -- $16$. \\
	\item Поделим единичный квадрат на $9$ квадратов со сторой $0.(3)$. Точек десять, поэтому хотя бы две попадут в один квадрат, тогда расстояние между ними будет $\le$ диагонали квадрата, которая равна $\frac{\sqrt(2)}{3}  \approx 0.4714 < 0.48$, \#, первый пункт есть. Второй пункт: поделим исходный квадрат на $4$ со стороной $0.5$, каждый из них можно покрыть кругом радиуса $0.5$, потому что диагональ квадрата $= \frac{1}{\sqrt(2)} < 1 =$ радиус круга. Если есть $10$ точек и $4$ круга, то по обобщённому принципу Дирихле хотя бы в один круг попадёт $3$ точки, \#. Доказано всё. \\
	\item Разобьём первые $2n$ чисел на $n$ пар вида $2i + 1, 2i + 2$, где $i : 0 \le i < n$. Тогда хотя бы из одной пары оба числа будут взяты, а $\forall x~(x, x + 1) = 1$, \#.\\
	\item ~\\
	\item Рассмотрим три горизонтальных прямых и их пересечения с девятью вертикальными. Каждое пересечение трёх горизонтальных с одной вертикальной -- это три точки, вариантов их раскраски $8$, то есть найдётся хотя бы одна пара вертикальных, дающих в пересечении с горизонтальными одинаковую раскраску. Среди трёх точек хотя бы две будут одного цвета, например, первая и третья. Тогда именно эти четыре точки (первая и третья на этих двух вертикальных) дадут прямоугольник с вершинами одного цвета. \#.  \\
	\item Рассмотрим префиксные суммы по модулю $n$. Если они все различны, то среди них есть $0$, ведь различных остатков по модулю $n$ тоже $n$. Тогда сумма на префиксе будет делиться на $n$ и мы победили. Если же нуля нет, то по принципу Дирихле есть две одинаковых префиксных суммы по модулю. Пусть это $i$-ая и $j$-ая префиксные суммы. Но тогда сумма на отрезке $(i, j]$ равна $0$ по модулю $n$, то есть делится на $n$, и мы снова победили. \#.
\end{enumerate}

\end{document}

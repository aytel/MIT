%
% Проверьте, что вы сохранили файл в кодировке utf8
%

\documentclass[12pt]{article} % 12 -- размер шрифта

\usepackage{cmap} % Чтобы можно было копировать русский текст из pdf
\usepackage[T2A]{fontenc}
\usepackage[russian,english]{babel}
\usepackage[utf8]{inputenc} % Проверьте, что кодировка файла -- тоже utf8
\usepackage{amsmath, amssymb} % Чтобы юзать математические символы

\usepackage{hologo} % Логотип LaTex
\usepackage[russian]{hyperref} % http ссылки на внешние источники

% Меняем размер листа, можно не менять
\textwidth=160mm
\hoffset=-15mm
\textheight=240mm
\voffset=-20mm

\newcommand{\Section}[1]{\section{#1}\vspace{-1.5em}\hspace*{\parindent}\unskip} % Фиксим багу с отступом в начале section, заодно пример собственных функций

\begin{document}

\Section{Примеры tex}

Это текст.

Формулы окружают долларами (\$) -- $a_1^5 + x^{100}_{i} + x_{i}^{100} - \frac{1}{2} = t^* - e'$.

Полезно заметить, что перед математическими стандарнтыми операциями вроде
логарифма стоит писать \texttt{\textbackslash}, например, $\log^2 n$, без него будет $log^2 n$.

{\bf Пример жирного.}

{\it Пример курсива.}

\texttt{Моноширинный шрифт. Удобно для C++ кода.}

\texttt{int main() \{ return 0; \}}

\def\t{\texttt} % Это наша собственная команда. Мы её в середине файла используем.
Мы же программисты? Можно создавать свои команды:\\

\t{Тестируем... Работает.}

Для перевода строки нужна пустая строка.
Здесь нет перевода строки.

Ещё строку переводить можно так: \texttt{``\textbackslash\textbackslash''}\\
Работает. Кстати, обратите внимания на кавычки, а тире --- оно длинное.

Как тексте могут быть формулы, так и в формулах текст!\\
$\frac{a \cdot \mbox{Я внутри дроби!} \cdot b}{\t{Я тоже! Как я сюда попал?}}$

\section{Заголовок: $1 = \Theta(3)$.}
\subsection{Подзаголовок: $2^n = \Omega(n^2)$, более того, $2^n = \omega(n^2)$.}
\subsubsection{Подподзаголовок: $n = O(n^2)$, $n^2 = \mathcal{O}(n^3)$, $\pi \in \mathbb{R}$} % Буквы можно окружать \mathcal, \mathbb

\Section{Списки.}

А еще тут есть пронумерованный списочек.
\begin{enumerate}
  \setlength{\parskip}{0pt} % Отступ перед списком
  \setlength{\itemsep}{0pt} % Отступ между строками списка
  \item Eins! Hier kommt die Sonne
  \item Zwei! Hier kommt die Sonne
  \item Drei! Sie ist der hellste Stern von allen
  \item Vier! Hier kommt die Sonne
  \item Funf! Hier kommt die Sonne
  \item Sechs! Hier kommt die Sonne
  \item Sieben! Sie ist der hellste Stern von allen
  \item Acht! Hier kommt die Sonne
\end{enumerate}

И не нумерованный
\begin{itemize}
  % Отступы можно не задавать руками, тогда они достаточно широкие
  \item Раз
  \item Двас
\end{itemize}

\pagebreak % Разрыв страницы

\Section{Про таблицы}

Можно использовать \texttt{array}. Он работает в math-mode (окружать долларами):

$$
\begin{array}{|r|c} % если сделать |r|c| будет более красиво!
  \hline
  \mbox{Номер группы} & \mbox{Имя повелителя}\\
  \hline
  1 & \mbox{Антон}\\
  2 & \mbox{Рома}\\
  \hline
  3 & \mbox{Сережа}\\
  \hline
\end{array}
$$

Можно использовать \texttt{tabular}:\\ % обратите внимание на \\

\begin{tabular}{|l|p{1cm}|p{1cm}|}
  \hline
    \multicolumn{3}{|c|}{Header} \\
  \hline
    $\int\limits_0^1xdx$ &first first first&second second second\\
  \hline
    p&q&r \\
  \hline
\end{tabular}\\	% обратите внимание на \\

\Section{Основные формулы}

$\alpha, \beta, \epsilon, \varepsilon, \phi, \varphi$

$\rightarrow, \Rightarrow, \to, \leftrightarrow, \Leftrightarrow$

$\sum, \int, \sum\limits_{i=1}^{n}i^2, \int\limits_{a}^{b}x dx, \prod, \cap, \cup, \circ, \bullet$

$\frac{\frac{a}{b}+c}{n^2+x^3+\sum y_i}$

\Section{Help}

Если хочется читать про tex, то

\url{https://en.wikibooks.org/wiki/LaTeX}

\url{https://en.wikibooks.org/wiki/LaTeX/Basics}

\url{https://en.wikibooks.org/wiki/LaTeX/Mathematics}

\url{https://en.wikibooks.org/wiki/LaTeX/Mathematics#List\_of\_Mathematical\_Symbols}

Гуглим книжку: \url{https://www.google.ru/?q=tex,Lvovskiy}

\vspace{1em}
\noindent \underline{\hbox to 1\textwidth{{ } \hfil{ } \hfil{ } }} % Горизонтальная линия
\vspace{1em}

На сегодня хватит. Удачного знакомство с \hologo{LaTeX}.

\end{document}
